%%% COULEURS %%%
\usepackage[table,svgnames]{xcolor}
\definecolor{nombres}{cmyk}{0,.8,.95,0}
\definecolor{gestion}{cmyk}{.75,1,.11,.12}
\definecolor{gestionbis}{cmyk}{.75,1,.11,.12}
\definecolor{grandeurs}{cmyk}{.02,.44,1,0}
\definecolor{geo}{cmyk}{.62,.1,0,0}
\definecolor{algo}{cmyk}{.69,.02,.36,0}
\definecolor{correction}{cmyk}{.63,.23,.93,.06}
\arrayrulecolor{couleur_theme} % Couleur des filets des tableaux

\usepackage[most]{tcolorbox} % Supprimé à cause de conflits avec ProfCollege
\tcbuselibrary{documentation}

%%%%%%%% Modification paramétrage pour footnotes
% Le paquet semble déjà appelé en amont, curieux qu'il n'y ait pas eu de clash
% avec l'appel ci-dessous mais seulement quand j'ai essayé de l'inclure avec l'option lualatex

% \usepackage{hyperref}
% Parametrages
\hypersetup{
    colorlinks=true,% On active la couleur pour les liens. Couleur par défaut rouge
    linkcolor=black,% On définit la couleur pour les liens internes
    % filecolor=magenta,% On définit la couleur pour les liens vers les fichiers locaux      
    urlcolor=black,% On définit la couleur pour les liens vers des sites web
    % pdftitle={Puissance Quatre},% On définit un titre pour le document pdf
    % pdfpagemode=FullScreen,% On fixe l'affichage par défaut à plein écran
    }
% Pour avoir les numérotations arabic y compris dans les environnements tcolorbox
\renewcommand{\thempfootnote}{\arabic{mpfootnote}}
%%%%%%%%

\usepackage{multicol} 
\usepackage{calc}
\usepackage{enumitem}
\usepackage{graphicx}
\usepackage{tabularx}

\setlength{\parindent}{0mm}
\renewcommand{\arraystretch}{1.5}
\renewcommand{\labelenumi}{\textbf{\theenumi{}.}}
\renewcommand{\labelenumii}{\textbf{\theenumii{}.}}
\setlength{\fboxsep}{3mm}

\setlength{\headheight}{14.5pt}

\setlength{\premulticols}{0pt}

\newcounter{ExoMA}

\newlength{\parindentMA}%
\setlength{\parindentMA}{\parindent}
\addtolength{\parindentMA}{30pt}

\usepackage{simplekv}

\setKVdefault[Theme]{Coopmath,Classiques=false,CANs=false}
\defKV[Theme]{Classique=\setKV[Theme]{Coopmath=false}\setKV[Theme]{Classiques}}
\defKV[Theme]{CAN=\setKV[Theme]{Coopmath=false}\setKV[Theme]{CANs}}

\usepackage{fancyhdr}
\fancyhead[C]{}
\fancyfoot{}

\def\bla{}

\NewDocumentCommand\Theme{o m m m m}{%
  \useKVdefault[Theme]%
  \setKV[Theme]{#1}%
  \ifboolKV[Theme]{CANs}{%
    \usepackage[left=1.5cm,right=1.5cm,top=2cm,bottom=2cm]{geometry}%
    \fancypagestyle{premierePage}{%
      \fancyhead[C]{\textsc{\ifx\bla#2\bla Course aux nombres\else#2\fi}}%
      \fancyfoot[R]{\scriptsize Coopmaths.fr -- CC-BY-SA}%
      \setlength{\headheight}{14.5pt}%
      \NewDocumentCommand\dureeCan{}{\ifx\bla#4\bla 9 minutes\else #4\fi}
      \NewDocumentCommand\titreSujetCan{}{\ifx\bla#3\bla\else\textbf{#3}\fi}
    }%
    \pagestyle{premierePage}
    \colorlet{couleur_theme}{black}%
    \colorlet{couleur_numerotation}{couleur_theme}%
    \let\EXO\EXOCan\let\endEXO\endEXOCan%
  }{%
    \renewcommand{\headrulewidth}{0pt}
    \renewcommand{\footrulewidth}{0pt}
    \pagestyle{fancy}
    \ifboolKV[Theme]{Classiques}{%
      \usepackage[left=0.65cm,right=0.65cm,top=2cm,bottom=1.5cm]{geometry}
      \let\EXO\EXOlibre\let\endEXO\endEXOlibre%
      \colorlet{couleur_theme}{black}%
      \colorlet{couleur_numerotation}{couleur_theme}%
      \fancyhead[C]{\bfseries #3}
      \fancyhead[L]{\bfseries #4}
      \fancyfoot[R]{#5}
    }{%
      \usepackage[left=1.5cm,right=1.5cm,top=4cm,bottom=2cm]{geometry}
      \theme{#2}{#3}{#4}{#5}%
      \let\EXO\EXOcoop\let\endEXO\endEXOcoop%
    }%
  }%
}%

\NewDocumentEnvironment{EXOCan}{m m +b}{%
  #3
}

\NewDocumentEnvironment{EXOcoop}{m m +b}{%
  \begin{tcolorbox}[%
    enhanced,
    breakable,
    colback=white,
    colframe=white,
    coltitle=black,
    title=\IfNoValueTF{#1}{}{\textmd{#1}},%
    overlay unbroken and first={%
      \IfNoValueTF{#2}{}{%
        \node[%
        fill=white,
        anchor=east,
        xshift=10pt,
        text=black
        ]
        at (frame.north east)
        {\scriptsize #2};
      }
      \node[anchor=west,xshift=-20pt,yshift=-15pt] at (frame.north west) {\stepcounter{ExoMA}\numb{\arabic{ExoMA}}};
    }
    ]
    #3
  \end{tcolorbox}
}

\NewDocumentEnvironment{EXOlibre}{m m +b}{%
  \begin{tcolorbox}[%
    enhanced,
    breakable,
    colback=white,
    colframe=white,%couleur_theme,
    coltitle=black,
    title=\stepcounter{ExoMA}\textbf{Exercice \arabic{ExoMA}},%
    ]
    \IfNoValueTF{#1}{}{#1\par}%
    #3
  \end{tcolorbox}%
}


%%% MISE EN PAGE %%%
\usepackage{setspace}
\usepackage{pgf}%
\usepackage{pgfplots}
\pgfplotsset{compat=1.18}
\usetikzlibrary{babel,arrows, arrows.meta,calc,fit,patterns,plotmarks,shapes.geometric,shapes.misc,shapes.symbols,shapes.arrows,shapes.callouts, shapes.multipart, shapes.gates.logic.US,shapes.gates.logic.IEC, er, automata,backgrounds,chains,topaths,trees,petri,mindmap,matrix, calendar, folding,fadings,through,positioning,scopes,decorations.fractals,decorations.shapes,decorations.text,decorations.pathmorphing,decorations.pathreplacing,decorations.footprints,decorations.markings,shadows}

\spaceskip=2\fontdimen2\font plus 3\fontdimen3\font minus3\fontdimen4\font\relax %Pour doubler l'espace entre les mots
\newcommand\numb[1]{ % Dessin autour du numéro d'exercice
  \begin{tikzpicture}[overlay,scale=.8]%,yshift=-.3cm
    \draw[fill=couleur_numerotation,couleur_numerotation](-.4,-0.1)rectangle($(-0.4,.8)+(2em,0)$);
    \draw (-.4,.8) node[white,anchor=north west,inner sep=0.5pt]{\bfseries EX}; 
    \draw[line width=.05cm,couleur_numerotation,fill=white] (0,0)--(.5,.5)--(1,0)--(.5,-.5)--cycle;
    \draw (0.5,0) node[anchor=center,couleur_numerotation]{\large\bfseries#1};
  \end{tikzpicture}
}

% echelle pour le dé
\def\globalscale {0.04}
% abscisse initiale pour les chevrons
\def\xini {3}

\newcommand\theme[4]{%
  \fancyhead[C]{%
    % Tracé du dé
    \begin{tikzpicture}[y=0.80pt, x=0.80pt, yscale=-\globalscale, xscale=\globalscale,remember picture, overlay,transform canvas={xshift=-0.5\linewidth,yshift=2.7cm},fill=couleur_theme]%,xshift=17cm,yshift=9.5cm
      %%%% Arc supérieur gauche%%%%
      \path[fill](523,1424)..controls(474,1413)and(404,1372)..(362,1333)..controls(322,1295)and(313,1272)..(331,1254)..controls(348,1236)and(369,1245)..(410,1283)..controls(458,1328)and(517,1356)..(575,1362)..controls(635,1368)and(646,1375)..(643,1404)..controls(641,1428)and(641,1428)..(596,1430)..controls(571,1431)and(538,1428)..(523,1424)--cycle;
      %%%% Dé face supérieur%%%%
      \path[fill](512,1272)..controls(490,1260)and(195,878)..(195,861)..controls(195,854)and(198,846)..(202,843)..controls(210,838)and(677,772)..(707,772)..controls(720,772)and(737,781)..(753,796)..controls(792,833)and(1057,1179)..(1057,1193)..controls(1057,1200)and(1053,1209)..(1048,1212)..controls(1038,1220)and(590,1283)..(551,1282)..controls(539,1282)and(521,1278)..(512,1272)--cycle;
      %%%% Dé faces gauche et droite%%%%
      \path[fill](1061,1167)..controls(1050,1158)and(978,1068)..(900,967)..controls(792,829)and(756,777)..(753,756)--(748,729)--(724,745)..controls(704,759)and(660,767)..(456,794)..controls(322,813)and(207,825)..(200,822)..controls(193,820)and(187,812)..(187,804)..controls(188,797)and(229,688)..(279,563)..controls(349,390)and(376,331)..(391,320)..controls(406,309)and(462,299)..(649,273)..controls(780,254)and(897,240)..(907,241)..controls(918,243)and(927,249)..(928,256)..controls(930,264)and(912,315)..(889,372)..controls(866,429)and(848,476)..(849,477)..controls(851,479)and(872,432)..(897,373)..controls(936,276)and(942,266)..(960,266)..controls(975,266)and(999,292)..(1089,408)..controls(1281,654)and(1290,666)..(1290,691)..controls(1290,720)and(1104,1175)..(1090,1180)..controls(1085,1182)and (1071,1176)..(1061,1167)--cycle;
      %%%% Arc inférieur bas%%%%
      \path[fill](1329,861)..controls(1316,848)and(1317,844)..(1339,788)..controls(1364,726)and(1367,654)..(1347,591)..controls(1330,539)and(1338,522)..(1375,526)..controls(1395,528)and(1400,533)..(1412,566)..controls(1432,624)and(1426,760)..(1401,821)..controls(1386,861)and(1380,866)..(1361,868)..controls(1348,870)and(1334,866)..(1329,861)--cycle;
      %%%% Arc inférieur gauche%%%%
      \path[fill](196,373)..controls(181,358)and(186,335)..(213,294)..controls(252,237)and(304,190)..(363,161)..controls(435,124)and(472,127)..(472,170)..controls(472,183)and(462,192)..(414,213)..controls(350,243)and(303,283)..(264,343)..controls(239,383)and(216,393)..(196,373)--cycle;
    \end{tikzpicture}
    \begin{tikzpicture}[line cap=round,line join=round,remember picture, overlay, shift={(current page.north west)},yshift=-8.5cm]
      \fill[fill=couleur_theme] (0,5) rectangle (21,6);
      \fill[fill=couleur_theme] (\xini,6)--(\xini+1.5,6)--(\xini+2.5,7)--(\xini+1.5,8)--(\xini,8)--(\xini+1,7)-- cycle;
      \fill[fill=couleur_theme] (\xini+2,6)--(\xini+2.5,6)--(\xini+3.5,7)--(\xini+2.5,8)--(\xini+2,8)--(\xini+3,7)-- cycle;  
      \fill[fill=couleur_theme] (\xini+3,6)--(\xini+3.5,6)--(\xini+4.5,7)--(\xini+3.5,8)--(\xini+3,8)--(\xini+4,7)-- cycle;   
      \node[color=white] at (10.5,5.5) {\LARGE \bfseries{ \MakeUppercase{#4}}};
    \end{tikzpicture}
    \begin{tikzpicture}[remember picture,overlay]
      \node[anchor=north east,inner sep=0pt] at ($(current page.north east)+(0,-.8cm)$) {};
      \node[anchor=west, fill=white, inner sep = 0pt] at ($(current page.north west)+(0.2,-2.3cm)$) {\footnotesize \bfseries{MathALÉA}};
    \end{tikzpicture}
    \begin{tikzpicture}[remember picture,overlay]
      \node[anchor=north east,inner sep=0pt] at ($(current page.north east)+(0,-.8cm)$) {};
      \node[anchor=east, fill=white] at ($(current page.north east)+(-2,-1.5cm)$) {\Huge \textcolor{couleur_theme}{\bfseries{\#}} \bfseries{#2} \textcolor{couleur_theme}{\bfseries \MakeUppercase{#3}}};
    \end{tikzpicture}
  }%
  \fancyfoot[R]{%
    \begin{tikzpicture}[remember picture,overlay]
      \node[anchor=south east] at ($(current page.south east)+(-2,0.25cm)$) {\scriptsize {\bfseries \href{https://coopmaths.fr/}{Coopmaths.fr} -- \href{http://creativecommons.fr/licences/}{CC-BY-SA}}};
    \end{tikzpicture}
    \begin{tikzpicture}[line cap=round,line join=round,remember picture, overlay,xscale=0.5,yscale=0.5, shift={(current page.south west)},xshift=35.7cm,yshift=-6cm]
      \fill[fill=couleur_theme] (\xini,6)--(\xini+1.5,6)--(\xini+2.5,7)--(\xini+1.5,8)--(\xini,8)--(\xini+1,7)-- cycle;
      \fill[fill=couleur_theme] (\xini+2,6)--(\xini+2.5,6)--(\xini+3.5,7)--(\xini+2.5,8)--(\xini+2,8)--(\xini+3,7)-- cycle;  
      \fill[fill=couleur_theme] (\xini+3,6)--(\xini+3.5,6)--(\xini+4.5,7)--(\xini+3.5,8)--(\xini+3,8)--(\xini+4,7)-- cycle;  
    \end{tikzpicture}
  }
  \fancyfoot[C]{}
  \colorlet{couleur_theme}{#1}
  \colorlet{couleur_numerotation}{couleur_theme}
  \def\iconeobjectif{icone-objectif-#1}
  \def\urliconeomethode{icone-methode-#1}
}%

\newcommand*\tikzfootMA{%
  \ifboolKV[Theme]{CANs}{
    \begin{tikzpicture}[remember picture,overlay,shift=(current page.north west)]
      \begin{scope}[x=\textwidth-\oddsidemargin,y=\textheight+5pt]
        \draw[correction,line width=4pt,dashed,dash pattern= on 10pt off 10pt,shift={(-\oddsidemargin,\topmargin)}] (0,0) rectangle (1,-1);
      \end{scope}
    \end{tikzpicture} 
  }{%
  \ifboolKV[Theme]{Classiques}{%
  \begin{tikzpicture}[remember picture,overlay,shift=(current page.south west)]
    \begin{scope}[x={(current page.south east)},y={(current page.north west)}]
      \draw[correction,line width=4pt,dashed,dash pattern= on 10pt off 10pt] ($(0,0)+(5mm,1cm)$) rectangle ($(1,1)+(-5mm,-1.75cm)$);
    \end{scope}
  \end{tikzpicture} 
  }{%
  \begin{tikzpicture}[remember picture,overlay,shift=(current page.south west)]
    \begin{scope}[x={(current page.south east)},y={(current page.north west)}]
      \draw[correction,line width=4pt,dashed,dash pattern= on 10pt off 10pt] ($(0,0)+(5mm,1.5cm)$) rectangle ($(1,1)+(-5mm,-3.75cm)$);
    \end{scope}
  \end{tikzpicture} 
  }%
  }
}

\newenvironment{Correction}{%
  \setcounter{ExoMA}{0}%
  \cfoot{\tikzfootMA}
}{}%

\newcommand\version[1]{
  \fancyhead[R]{
    \begin{tikzpicture}[remember picture,overlay]
    \node[anchor=north east,inner sep=0pt] at ($(current page.north east)+(-.5,-.5cm)$) {\large \textcolor{couleur_theme}{\bfseries V#1}};
    \end{tikzpicture}
  }
}

\usepackage[french]{babel}