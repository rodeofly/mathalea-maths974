\textbf{\large Exercice 1 \hfill 5 points}

\bigskip

Une ancienne légende raconte que le jeu d'échecs a été inventé par un vieux sage. 

Son roi voulut le remercier en lui accordant n'importe quel cadeau en récompense. 

Le vieux sage demanda qu'on lui fournisse un peu de riz pour ses vieux jours, et plus précisément qu'on place :

un grain de riz sur la première case du jeu qu'il venait d'inventer, puis deux grains de riz sur la case suivante, puis quatre grains de riz sur la troisième case, et ainsi de suite, en doublant le nombre de grain de riz entre une case et la suivante, et ce jusqu'à la 64\ieme{} case (puisqu'un plateau de jeu d'échecs comporte 64 cases).

\begin{center}
\psset{unit=0.6cm}
\def\xmin {0}   \def\xmax {8}
\def\ymin {0}   \def\ymax {8}
\begin{pspicture}(\xmin,\ymin)(\xmax,\ymax)
\psgrid[subgriddiv=1,  gridlabels=0, gridcolor=black]
\psset{dotscale=0.8,linecolor=blue}
\psdots(0.5,7.5)
\psdots(1.2,7.2)(1.8,7.8)
\psdots(2.2,7.2)(2.8,7.8)(2.2,7.8)(2.8,7.2)
\psdots(3.2,7.2)(3.8,7.8)(3.2,7.8)(3.8,7.2)(3.2,7.5)(3.8,7.5)(3.5,7.8)(3.5,7.2)
\end{pspicture}
\end{center}


 
On note $u_1$ le nombre de grains de riz présents sur la première case, $u_2$ le nombre de grains sur la deuxième case, et ainsi de suite jusqu'à la 64\ieme{} case.

\begin{enumerate}
\item Déterminer $u_1$, $u_2$, $u_3$, $u_4$ et $u_5$.

\item Exprimer, pour tout entier naturel $n$ non nul, $u_{n+1}$ en fonction de $u_n$.

\item En déduire la nature de la suite $(u_n)$ et en préciser les éléments caractéristiques.

Exprimer, pour tout entier naturel $n$ non nul, $u_n$ en fonction de $n$.

\item Calculer le nombre de grains de riz qui doivent être disposés sur le plateau pour satisfaire à la demande du vieux sage.
\end{enumerate}

\begin{multicols}{2}
\begin{enumerate}
\setcounter{enumi}{4}
\item  On veut écrire une fonction en langage Python qui détermine à partir de quelle case, le vieux sage disposera d'au moins $R$ grains de riz. 

Une ébauche de cette fonction est donnée ci-contre.

Recopier et compléter cette fonction afin qu'elle renvoie le résultat désiré.
\end{enumerate}

\columnbreak

\begin{center}
\fbox{\tt
\begin{tabular}{l}
def {\blue nb\_cases} ({\bf R}) :\\
\hspace*{0.5cm}{\bf case} = 1\\
\hspace*{0.5cm}{\bf u} = 1\\
\hspace*{0.5cm}{\bf somme} = {\bf u}\\
\hspace*{0.5cm}while {\bf somme} ...... :\\
\hspace*{1cm}{\bf u} = ...\\
\hspace*{1cm}{\bf somme} = ...\\
\hspace*{1cm}{\bf case} = {\bf case} + 1\\
\hspace*{0.5cm}return {\bf case}
\end{tabular}
}
\end{center}
\end{multicols}


